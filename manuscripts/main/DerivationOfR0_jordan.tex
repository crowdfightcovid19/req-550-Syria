\documentclass{article}
\usepackage[utf8]{inputenc}
\usepackage{amsmath}
\usepackage{mathtools}

\title{Derivation of R0}

\begin{document}

\maketitle

\subsection*{Reference and general idea}

The derivation of R0 for a SIR type model following description of 'Mathematical Tools for Understanding Infectious Disease Dynamics' written by Odo Diekmann, Hans Heesterbeek and Tom Britton. 
See chapter 7, section 2 'Next-generation matrix for compartmental systems'. 
We use 4 steps to find the basic reproduction number of a system.
1) Define the infected subsystem, i.e. all equations that include compartments where individuals can get infected. 
2) Linearize the subsystem in the disease free equilibrium. 
3) Find the next generation matrix (NGM) with large domain ($K_L$) by writing the linear system as: $\dot{x} = (T + \Sigma) x$, here $x$ is a vector containing all states where individuals can get infected, $T$ contains all terms corresponding to transmission and $\Sigma$ all terms corresponding to duration of stay in a compartment. Then: $NGM = -T \Sigma^{-1}. $
4) The reproduction number is now the dominant eigenvalue of the NGM. 
We also reference the description of R0 in metapopulations in Philipps S, Rossi D. Mathematical Models of Infectious Diseases: Two-Strain Infections in Metapopulations, and the methodology for calculating R0 used by the authors in Gatto M, Bertuzzo E, Mari L, Miccoli S, Carraro L, Casagrandi R, et al. Spread and dynamics of the COVID-19 epidemic in Italy: Effects of emergency containment measures. PNAS. 2020 May 12;117(19):10484–91.


\subsection*{Infected subsystem}

The infected subsystem is the following: 
\\
\begin{gather*}
    \dot{E}_i =  \lambda_i S_i - \delta_E E_i \\
    \dot{P}_i = \delta_E E_i - \delta_p P_i \\
    \dot{A}_i = (1-f) \delta_p P_i - \gamma_A A_i \\
    \dot{I}_i = f \delta_p P_i - ((1-g_i-h_i) \gamma_{I} + h_i \eta + g_i \alpha) I_i \\   
    \dot{H}_i = h_i \eta I_i - \gamma_H H_i   
\end{gather*}
\\

Where, 
\\
\begin{gather*}
   \lambda_i = \sum_{j = 1}^n C_{ij} \frac{\beta_P P_j + \beta_A A_j + \beta_I I_j + \beta_H H_j}{N_j}
\end{gather*}
\\

\subsection*{Contact matrix}

$C_{ij}$ in the above equation is the contact matrix, the probability that individuals who are from population class $i$ come into contact with individuals who are from population class $j$. More specifically, if all population classes had the same size, if an individual from class $i$ were to make a single contact, $C_{ij}$ measures the probability that it would be with an individual from class $j$. $i$ and $j$ can overlap and rows of the matrix sum to 1. $C_{ij}$  is dimensionless and is independent of both  absolute number of contacts and relative population size of each class. \\

To illustrate with our model of 5 population classes, we construct the contact matrix from the product of two matrices, an unscaled contact matrix we will denote $\boldsymbol{U_{ij}}$ and a scaling matrix we will deonte $\boldsymbol{S}$. $\boldsymbol{U_{ij}}$ is a symmetric matrix describing the inter-class contact rate, $\epsilon_{i,j}$ relative to intra-class contact, set to a value of 1. The diagonals of the matrix quantify intra-class contact while the off-diagonals quantify inter-class contact.

\begin{gather}
   \boldsymbol{U_{ij}} = 
   \begin{bmatrix}
1 & \epsilon_{1,2} & \epsilon_{1,3} & \epsilon_{1,4} & \epsilon_{1,5} \\
\epsilon_{1,2} & 1 & \epsilon_{2,3} & \epsilon_{2,4} & \epsilon_{2,5} \\
\epsilon_{1,3} & \epsilon_{2,3} & 1 & \epsilon_{3,4} & \epsilon_{3,5} \\
\epsilon_{1,4} & \epsilon_{2,4} & \epsilon_{3,4} & 1 & \epsilon_{4,5} \\
\epsilon_{1,5} & \epsilon_{2,5} & \epsilon_{3,5} & \epsilon_{4,5} & 1 
\end{bmatrix}
\end{gather}

$\boldsymbol{S}$ is a diagonal matrix whose nonzero values are the inverse of the sum of each row in $\boldsymbol{U_{ij}}$. $\boldsymbol{S} =$ 


\noindent
\[
\begin{bmatrix}
\frac{1}{1+\epsilon_{1,2}+\epsilon_{1,3}+\epsilon_{1,4}+\epsilon_{1,5}} & 0 & 0 & 0 & 0 \\
0 & \frac{1}{\epsilon_{1,2} + 1 + \epsilon_{2,3} + \epsilon_{2,4} + \epsilon_{2,5}} & 0 & 0 & 0 \\
0 & 0 & \frac{1}{\epsilon_{1,3}+\epsilon_{2,3}+1+\epsilon_{3,4}+\epsilon_{3,5}} & 0 & 0 \\
0 & 0 & 0 & \frac{1}{\epsilon_{1,4}+\epsilon_{2,4}+\epsilon_{3,4}+1+\epsilon_{4,5}} \\
0 & 0 & 0 & 0 & \frac{1}{\epsilon_{1,5}+\epsilon_{2,5}+\epsilon_{3,5}+\epsilon_{4,5}+1} 
   \end{bmatrix}
\]


$\boldsymbol{S}$ scales $\boldsymbol{U_{ij}}$ such that each value in $C_{ij} \in [0,1]$ and $\boldsymbol{S}\boldsymbol{U_{ij}}=C_{ij}$. This gives us a dimensionless measure of contact probability independent of both population structure and other parameters, that is easy to interpret and modify. For additional information on this method, see page 7 of Philipps S, Rossi D. Mathematical Models of Infectious Diseases: Two-Strain Infections in Metapopulations.\\

\subsection*{Linearize the subsystem}

Returning to our infected subsystem, we linearize it and write it in the form of $\dot{x} = (T + \Sigma) x$, this gives us:

\begin{gather}
 T =
  \begin{bmatrix}
   \boldsymbol{0} & \boldsymbol{\beta_P} \boldsymbol{\Theta} & 
   \boldsymbol{\beta_A} \boldsymbol{\Theta} &
   \boldsymbol{\beta_I} \boldsymbol{\Theta} &
   \boldsymbol{\beta_H} \boldsymbol{\Theta} \\
   \boldsymbol{0} & \boldsymbol{0} & \boldsymbol{0} & \boldsymbol{0} & \boldsymbol{0} \\
   \boldsymbol{0} & \boldsymbol{0} & \boldsymbol{0} & \boldsymbol{0} & \boldsymbol{0} \\
   \boldsymbol{0} & \boldsymbol{0} & \boldsymbol{0} & \boldsymbol{0} & \boldsymbol{0} \\
   \boldsymbol{0} & \boldsymbol{0} & \boldsymbol{0} & \boldsymbol{0} & \boldsymbol{0}
   \end{bmatrix}
\end{gather}
\begin{gather}
 \Sigma =
  \begin{bmatrix}
   - \delta_E \boldsymbol{I} & \boldsymbol{0} & \boldsymbol{0} & \boldsymbol{0} & \boldsymbol{0} \\
   \delta_E \boldsymbol{I}& -\delta_p \boldsymbol{I} & \boldsymbol{0} & \boldsymbol{0} & \boldsymbol{0} \\
   \boldsymbol{0} & (1-f) \delta_p \boldsymbol{I} & -\gamma_A \boldsymbol{I} & \boldsymbol{0} & \boldsymbol{0} \\
   \boldsymbol{0} & f \delta_p \boldsymbol{I} & \boldsymbol{0} & - ((1-g_i-h_i) \gamma_{I} + h_i \eta + g_i \alpha) \boldsymbol{I} & \boldsymbol{0} \\
   \boldsymbol{0} & \boldsymbol{0} & \boldsymbol{0} & h_i \eta \boldsymbol{I} & -\gamma_H \boldsymbol{I}
   \end{bmatrix}
\end{gather}

Where $\boldsymbol{I}$ and $\boldsymbol{0}$ are the identity and null matrices of size n. \\

\subsection*{Define the $\beta$ matrices}

$\boldsymbol{\beta_P}$, $\boldsymbol{\beta_A}$, $\boldsymbol{\beta_I}$, and $\boldsymbol{\beta_H}$ are diagonal matrices whose nonzero elements are the transmission parameters, $\beta$, for each population class for that specific infectious compartment. To illustrate using the 5 population classes in our model:
\\
\begin{gather}
   \boldsymbol{\beta_i} (i \in \{P,A,I,H\}) = 
   \begin{bmatrix}
   \beta_{i,age1} & 0 & 0 & 0 & 0 \\
   0 & \beta_{i,age2nocomorbid} & 0 & 0 & 0 \\
   0 & 0 & \beta_{i,age2comorbid} & 0 & 0 \\
   0 & 0 & 0 & \beta_{i,age3nocomorbid} & 0 \\
   0 & 0 & 0 & 0 & \beta_{i,age3comorbid}
   \end{bmatrix}
\end{gather}
\\

In our model, children on average have twice as many contacts as adults and the elderly. Therefore, 
\\
\begin{gather*}
   \beta_{i,age1} = 2\beta_{i,age2nocomorbid} =  2\beta_{i,age2comorbid} = 2\beta_{i,age3nocomorbid} = 2\beta_{i,age3comorbid}
\end{gather*}\\
\newpage

\subsection*{Define the $\Theta$ matrix}

Let $\boldsymbol{\Theta} = \boldsymbol{N} \boldsymbol{C} \boldsymbol{N}^{-1}$\\

Where $\boldsymbol{N}$ is a diagonal matrix whose nonzero elements are the relative sizes of each population class, $N_i$, and $\boldsymbol{C}$ is the contact matrix $C_{ij}$. \\

The dimensionless contact matrix needs to be adjusted for the relative population size of each population class that come into contact in our model. We do this by taking the product of $\boldsymbol{N}$ and $\boldsymbol{C}$ and then taking the product of the resulting matrix and $\boldsymbol{N}^{-1}$. Only taking the product of $\boldsymbol{N}$ and $\boldsymbol{C}$ or $\boldsymbol{C}$ and $\boldsymbol{N}^{-1}$ will yield a $\boldsymbol{\Theta}$ matrix that will either inflate or deflate our derived $R_0$.\\

To test, the spectral radius of $\boldsymbol{\Theta}$ absent any other parameters should be 1; this is analogous to a steady state disease system in which new cases enter and leave the infected state at the same rate. Using a perfectly well-mixed population and our model's population structure, we find $\rho(\boldsymbol{C} \boldsymbol{N}^{-1}) > 1$, $\rho(\boldsymbol{N} \boldsymbol{C}) < 1$, and $\rho(\boldsymbol{N} \boldsymbol{C} \boldsymbol{N}^{-1}) = 1$. For further details on this method of computing a $\boldsymbol{\Theta}$ matrix, see the supplementary materials in Gatto M, Bertuzzo E, Mari L, Miccoli S, Carraro L, Casagrandi R, et al. Spread and dynamics of the COVID-19 epidemic in Italy: Effects of emergency containment measures. PNAS. 2020 May 12;117(19):10484–91.\\  

\subsection*{Finding the NGM and deriving $R_0$}

Returning to $\Sigma$, we find its inverse $\Sigma^{-1} = $
\\
\noindent
\[
  \begin{bmatrix}  
   - \frac{1}{\delta_E} \boldsymbol{I} & \boldsymbol{0} & \boldsymbol{0} & \boldsymbol{0} & \boldsymbol{0} \\
   -\frac{1}{\delta_p} \boldsymbol{I} & -\frac{1}{\delta_p} \boldsymbol{I} & \boldsymbol{0} & \boldsymbol{0} & \boldsymbol{0} \\
   -\frac{(1-f)}{\gamma_A} \boldsymbol{I} & -\frac{(1-f)}{\gamma_A} \boldsymbol{I} & -\frac{1}{\gamma_A} \boldsymbol{I} & \boldsymbol{0} & \boldsymbol{0} \\
   -\frac{f}{((1-g_i-h_i) \gamma_{I} + h_i \eta + g_i \alpha)} \boldsymbol{I} & -\frac{f}{((1-g_i-h_i) \gamma_{I} + h_i \eta + g_i \alpha)} \boldsymbol{I} & \boldsymbol{0} & -\frac{1}{((1-g_i-h_i) \gamma_{I} + h_i \eta + g_i \alpha)} \boldsymbol{I} & \boldsymbol{0} \\
   -\frac{fh_i \eta}{((1-g_i-h_i) \gamma_{I} + h_i \eta + g_i \alpha)\gamma_H} \boldsymbol{I} & -\frac{fh_i \eta}{((1-g_i-h_i) \gamma_{I} + h_i \eta + g_i \alpha)\gamma_H} \boldsymbol{I} & \boldsymbol{0} & -\frac{h_i \eta}{((1-g_i-h_i) \gamma_{I} + h_i \eta + g_i \alpha)\gamma_H} \boldsymbol{I} & -\frac{1}{\gamma_H} \boldsymbol{I} 
   \end{bmatrix}
\]
\\
The NGM with large domain can now be found by $K_L = - T \Sigma^{-1}$, however, as we know that each individual that gets infected first will go to the $E$ compartment we can instead calculate the NGM with small domain that only consists of the $E$ compartment.
We do this by removing from $T$ the rows that correspond to the other compartments and from $\Sigma^{-1}$ the columns. 
We then find: 

\begin{gather*}
\boldsymbol{K} = \frac{1}{\delta_P} \boldsymbol{\beta_P} \boldsymbol{\Theta} + \frac{(1-f)}{\gamma_A} \boldsymbol{\beta_A} \boldsymbol{\Theta} + \frac{f}{((1-g_i-h_i) \gamma_{I} + h_i \eta + g_i \alpha)} \boldsymbol{\beta_I} \boldsymbol{\Theta} + \frac{fh_i \eta}{((1-g_i-h_i) \gamma_{I} + h_i \eta + g_i \alpha)\gamma_H} \boldsymbol{\beta_H} \boldsymbol{\Theta}
\end{gather*}

Since $R_0$ is the spectral radius of the NGM $K$,\\

\begin{gather}
R_0 = \rho(\boldsymbol{K})
\end{gather}

\end{document}
