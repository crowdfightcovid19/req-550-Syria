%% LyX 2.3.2 created this file.  For more info, see http://www.lyx.org/.
%% Do not edit unless you really know what you are doing.
\documentclass[english]{article}
\usepackage[T1]{fontenc}
\usepackage[latin9]{inputenc}
\usepackage{geometry}
\geometry{verbose,tmargin=2cm,bmargin=2.5cm,lmargin=2.3cm,rmargin=2cm}
\usepackage{verbatim}

\makeatletter
%%%%%%%%%%%%%%%%%%%%%%%%%%%%%% Textclass specific LaTeX commands.
\newcommand{\lyxaddress}[1]{
	\par {\raggedright #1
	\vspace{1.4em}
	\noindent\par}
}

%%%%%%%%%%%%%%%%%%%%%%%%%%%%%% User specified LaTeX commands.
\usepackage{lipsum}

\makeatother

\usepackage{babel}
\begin{document}
\title{Empowering the crowd: Feasible strategies \\ to minimize the spread
of COVID-19 \\ in informal settlements}
\author{Authors names TBA}
\maketitle

\lyxaddress{Add your affiliation here.}
\begin{abstract}
Background, Methods, Findings, Interpretation, and Funding. \lipsum[1]
\end{abstract}

\part*{Introduction}

There is still a substantial number of low-income countries in which
the COVID-19 pandemic has not reached its fullest potential. %
\begin{comment}
APG: the following is just an example, we are looking for a comparison
that allow us to warn that this work is important because in many
places in which our work is relevant, the virus has not appeared yet
\end{comment}
{} In the beginning of August 2020, the number of cases reported in
African countries per inhabitant was XXX. These numbers are still
far from those reported in (XXX e.g. Europe). The fragile health-systems
in these countries {[}Ref{]}, in particular regarding critical care
{[}Ref{]}, and the often lack of access to basic services such as
water sanitation {[}Ref{]}, may have major implications in the spread
of the disease {[}Ref{]}.

These figures become even worse in regions immerse in armed conflicts,
in which informal settlements of Internally Displaced Population (IDP)
are widespread. In these settlements, the number of inter-personal
contacts is very high, with people often living in overcrowded tents.
For the region we take as a reference in this study, the North West
of Syria, XX\% of the households are IDP living in tents with XX people
on average {[}Verify and Ref{]}. Moreover, populations in these regions
often have a high proportion of individuals suffering from comorbidities.
In the NW Syria, 10\% of the population has a chronic disease, with
17\% of them having no access to medicines {[}REF Reach doc{]}, and
it is often dependent on humanitarian organizations {[}data?{]}. At
a regional scale, the political instability in areas affected by armed
conflicts hinders the efficiency of coordinated actions, and military
operations often prompt the exodus of the population. These movements
could facilitate the transmission of the disease, also making tracing
interventions unfeasible. At the local scale, a large number of informal
camps (not registered by the WHO {[}APG, please precise{]}) have limited
or no management {[}REF Reach{]}, again precluding the implementation
of interventions to limit the spread of the virus.

To investigate feasible interventions in informal settlements (hereafter
named ``camps''), we considered a Susceptible-Exposed-Infectious-Recovered
model in which the population is divided into classes, reflecting
the age-structure and the existence of comorbidities {[}Ref{]}. We
use this model to propose a number of interventions aimed at reducing
the contacts within and between population classes in general, and
with symptomatic individuals in particular. In order to incorporate
these interventions, the model includes an explicit parametrization
of the contacts that the populations have per day, an approximation
that was previously used to model the effect of similar interventions
in minimizing the spread of the COVID-19 in African cities {[}Zandvoort{]}.
We build upon this approximation making a more explicit representation
of the contacts, and of other parameters in the model. We paid special
attention on the assumptions that the different interventions would
have under the living conditions that exist in these camps.

We considered a worst-case scenario in which there is no access to
any health-care facility. We modelled previously proposed interventions
{[}Zandvoort{]} such as self-distancing, isolation of symptomatic
individuals and the creation of a 'safety zone' in which more vulnerable
population becomes less exposed to the virus. We elaborated in more
detail previous models {[}Ref, Zandvoort{]} to take into account the
micro-dynamics of the interactions, considering questions such as
the time than an individual takes to recognize the symptoms before
self-isolating, the effect of having carers to attend isolated individuals,
or the existence of a buffering zone in which exposed and protected
population classes could interact under certain rules, among other
details. Since empowering the local communities to understand how
to control the virus is possibly the most (and perhaps unique) effective
way to minimize its spread, the details we modelled become of outmost
importance for a realistic implementation of the interventions.

\section*{Methods}

\subsection*{The model}

We considered a discrete-time stochastic model simulating the spread
of the virus in a single camp. The model is divided in compartments,
containing individuals at different possible epidemiological stages
along the progression of the disease (see Supplementary Material).
The model additionally considers that the population is divided in
different classes depending on the age-structure and comorbidities.
The simulation starts with a population of susceptible individuals
and then one individual becomes exposed. The progression of the disease
in the individual goes through a series of infectious stages: a presymptomatic
stage, and then either a symptomatic or asymptomatic stage, eventually
infecting other suceptible individuals. Individuals are no longer
infectious if they recover or if they die. The simulations run for
365 days, and we verified that a steady state was always reached before
the limit of the simulation was achived. We did not considered any
additional entry of infectious individuals in the camp during the
simulation. Births and deaths due to other causes were not considered
in the model, since populations changes due to the disease are much
larger than those related to these processes, provided that there
are no conflicts during the period simulated.

\subsection*{Population structure}

We considered data from the NW of Syria as a reference to parametrize
the models {[}SOURCE{]}. The population size of informal camps was
log-normally distributed, with a mean size of 600 individuals. We
simulated population sizes of 500, 1000 and 2000 individuals. Unless
stated otherwise, the results presented refer to populations of 2000
individuals, because interventions tend to be less effective in larger
camps. The original data was structured in three population ages:
age 1 (0-13 years old), age 2 (14-50 yrs.) and age 3 (>50 yrs). For
the classes correspondent to age 2 and 3, we additionally considered
two subclasses comprising health individuals and individuals with
comorbidities, respectively. Therefore, in the absence of interventions,
we considered 5 population classes, whose proportions are shown in
Table {[}create Table{]}.

\subsection*{Epidemiological severity assumptions}

An important question to consider in the model is the access to health
care. In the case of the region we are considering as a reference,
NW Syria, there are available XXX beds in hospitals (XXX UCIs) and
XX ventilators for 4.5 million people {[}Ref{]}. Basic estimations
predicted a collapse of the health facilities after 8 weeks of the
outbreak {[}{]}. Given the additional difficulties that people living
in the camps has to access these facilities {[}how to show this?{]},
we considered a worst-case scenario in which individuals will not
have access to health care. As a consequence, we assumed that symptomatic
individuals requiring an ICU would die. More problematic is to estimate
the fate of those individuals that would have symptoms with a severity
requiring hospitalization but not necessarily ICU care, since in the
absence of hospitalization their fate is uncertain. To address this
point, and departing from previous models {[}Refs{]}, we considered
a hospitalization compartment. Since, as we said, hospitalization
is unfeasible, these individuals are considered to stay in the camp
and being infectious. Hence, the role of this compartment is to account
for a longer infectious period for these individuals. This compartment
will also help us to model more realistically some interventions,
for example noting that the severity of the symptoms for these individuals
is incompatible with self-isolation. From this compartment, we simulated
two possible extreme scenarios, one in which all these individuals
would recover, and another in which these individuals will all die.
This strategy allowed us to estimate upper and lower bounds for the
different variables. In the simulations presented in the Main Text,
we considered a worst-case scenario, namely the one in which all individuals
die.

Another consideration comes from the estimation of the fraction of
the population that will suffer severe or very severe symptoms (parameters
$h_{i}$ and $g_{i}$, see Table {[}Table{]}). These parameters are
class-specific, but we gathered this data from studies in countries
in which the living conditions are more favourable {[}Refs{]}, and
in which there is access to health care. Following previous work {[}Zandvoort{]},
we considered that in, our setting, the fraction of individuals showing
severe symptoms will be larger, and we implemented this fact mapping
the population-classes in our model to population-classes 10 years
older in the studies we took as a reference.

\subsection*{Transmissibility assumptions}

In our model, individuals from the presymptomatic, asymptomatic, symptomatic
and hospitalized compartments are equally infectious. Although asymptomatic
individuals are often considered less infectious {[}Ref{]}, the fraction
of asymptomatic we considered was of 16\% {[}Ref{]}, which might be
an underestimation after recent findings reporting up to 49\% {[}Ref{]},
and hence we opt to considered them equally infectious than symptomatic
ones. The length of the time intervals that individuals spend in each
compartment was retrieved from literature (see Table {[}Table{]} and
Supplementary Material).

To model the contacts between individuals we considered that the number
of contacts between individuals of different classes was the product
of the mean number of contacts that an individual has per day ($\bar{c}_{i}$,
see Supplementary Materia) times the probability of finding an individual
of other class, which is proportional to the proportion of that class
in the total population. The $\bar{c}_{i}$ values were estimated
from direct conversations with people managing the camps, although
it was not possible to perform a formal survey. We estimated that
individuals of age 1 have 25 contacts per day on average, those of
age 2 have 15 contacts per day, and we fix to 10 contacts per day
those of age 3.

The probability of getting infected after a contact with an infectious
individual, $\tau$, was estimated considering a Gaussian distribution
of the basic reproduction number $R_{0}$ with mean 4 and 99\% CI=(3--5).
These values were a compromise between values reported in the literature
from cities including favelas, ranging from $R_{0}$=2.77, 3.44 in
Abuja and Lagos, respectively (Nigeria, {[}Ref{]}); $R_{0}$ =3.3
in Buenos Aires (Argentina, {[}Ref{]}), to Rohinya's refugee camps
that were estimated to be as high as $R_{0}$=5 {[}Ref{]}. The probability
distribution for $\tau$ was estimated randomly generating a value
for $R_{0}$, and dividing the value by the real part of the main
eigenvalue of the Next Generation Matrix (see Supplementary Material).

\subsection*{Interventions}

\begin{comment}
APG: A table similar to Table 1 in Zandvoort et al. would be great
\end{comment}


\subsubsection*{Self-distancing}

We considered a situation in which all the population reduce the mean
number of contacts per day by a certain amount (20\% and 50\%). Since
the mean number of people per tent in a camp is 7 {[}verify{]}, we
considered that the number of contacts per day cannot be reduced by
more than a 50\%. Note that, for an adult, this would mean having
7.5 contacts per day.

\subsubsection*{Buffering zone}

When individual self-distancing cannot be achieved, a possible solution
is to split the population in groups, and apply self-distancing measures
to individuals belonging to different groups. We name the zones in
which interactions between population groups occur ``buffering zones''.
We assume that these zones are always open spaces, not occupied by
more than 4 individuals wearing masks, and in which 2 meters of distance
between individuals of different groups is guaranteed. This setting
allowed us to assume a reduction in the probability of infection per
contact of an 80\% {[}Ref{]}. An additional intervention we considered
in some simulations, is that symptomatic individuals cannot access
to a buffering zone.

\subsubsection*{Safety zone}

In this intervention, it is considered that the camp is divided in
two areas: a safety zone, in which more vulnerable people lives (that
we will refer to as ``green'' zone following previous studies {[}Ref{]}),
and an exposed (orange) zone with the remainder population. In the
simulations, it is always considered that the first exposed individual
belongs to the orange zone. Contrary to the nomenclature followed
in other studies (see e.g. {[}Ref{]}), we avoided the use of the term
``shielding'' for this intervention, since it may lead to a (misleading)
picture in which it is believed that the vulnerable population is
isolated into a closed space, such as a separated building. Such intervention
would require additional assumptions on how contacts occur in such
a space %
\begin{comment}
APG: Are we aware of any study that address this point? there has
been much controversy in the media around this
\end{comment}
. In our simulations, we instead assumed that the camp is split in
two separated areas, and that the number of contacts between individuals
living in different areas is reduced, while the living conditions
within both areas remain otherwise the same. Therefore, the number
of contacts per day and individual does not change unless self-distancing
is implemented. In practice, this means that, for every individual,
reducing the number of contacts with individuals living in the other
zone, implies an increase in the number of contacts with the individuals
in their own zone (see Supplementary Material). This is important
to investigate undesired side-effects that the split may have.

We also considered that individuals living in the green zone cannot
leave that area, which means that supplies for this population should
be provided by individuals in the exposed zone, and hence the contacts
between both populations never vanishes. However, we considered that
this interaction will always occur in a buffering zone, and we assumed
that each individual in the green zone can have a maximum number of
contacts per week with individuals of the orange zone: either 10 or
2 contacts per week for each individual.

An important social question to be addressed in this intervention
is how population will be distributed in the two zones, since different
partitionings may have a different reception in the population. We
considered the following isolation scenarios: i) elderly population
only (age 3); ii) elderly population and adults (age 2) with comorbidities;
and iii) elderly, adults with comorbidities, and other adults (e.g.
spouses or carers) and their young kids (< 13 yrs. and not exceeding
40\% of the population in the green zone). For the scenario iii) we
considered, in turn, three additional possibilities, in which the
total population in the green zone represent 20\%, 25\% or a maximum
of 30\% of the total population.

\subsubsection*{Lockdown of the safety zone}

An additional measure that can incorporated once the safety zone is
implemented, is a further reduction in the number of contacts occuring
in the buffering zone after the first symptomatic case in the orange
zone is identified. We refer to this reduction as ``lockdown'' (of
the safety zone). Again, since supplies for people living in the green
zone cannot be disrupted, we assumed that the number of contacts per
week and individual is reduced by a 50\% or a maximum of a 90\% (see
Table XXX)

\subsubsection*{Self-isolation}

A challenge in informal settlements is how self-isolation can be interpreted
when households have a single, and often small space. Moreover, water
should be collected at specific locations, there are communal latrines
and the scarcity of food supplies do not allow for the isolation of
whole families. We considered the possibility that individuals showing
symptoms self-isolate in individual tents in dedicated spaces of the
camps, or next to the tents of their relatives. To simulate a more
realistic scenario, we considered that there is a minimum time for
an individual to recognize his/her own symptoms, within a period ranging
from 12 to 48 hours (see Table XXX). In addition, we considered that
there are a number of carers dedicated to supply food and water to
isolated individuals, interacting in a buffering zone. In Supplementary
Material we show that considering one carer per individual having
one contact per day, we account for both the positive effects of isolation
and the negative effects derived from the contact with carers, since
the probability of infection with the rest of the population cannot
be neglected. In this respect, we considered that people moving to
a severely symptomatic stage (hospitalized compartment) becomes fully
infectious. The rationale behind this choice is that this individuals
require a more intensive care, and hence it is expected that the rules
imposed for a buffering zone cannot be longer fulfilled. We modelled
this intervention for an increasing number of tents from 10 to an
unlimited number (see Table XXX).

\subsubsection*{Evacuation}

The last intervention refers to the possibility of evacuating severely
symptomatic individuals, i.e. those in the hospitalization compartment.
This measure does not change the fate of the evacuated individuals.
Hence, we assume that the evacuation will not be to a hospital but
to some sort of isolation center %
\begin{comment}
APG: Chamsy, I think it would be great to document here that this
is a possibility considered by policy-makers.
\end{comment}
. In this way, the measure will reduce the infectiousness of these
individuals to zero.

\subsection*{Analysis of the interventions}

For each implementation of the interventions, we run 500 simulations,
analysing results arising from the incorporation of an intervention
with respect to a set of simulations in which the intervention was
absent. The main variables we considered in the comparisons were the
fraction of casualties in the population, the case fatality rates
(CFR), the fraction of population recovered and the time at which
it is observed a peak in the number of symptomatic. We focused on
the symptomatic population, because it is the most informative measure
in settings in which there is no access to other tests. We additionally
considered the fraction of simulations in which at least one death
is observed as a proxy for the probability of outbreak.%
\begin{comment}
Note that interventions will affect CFR only when interventions have
a differential effect among population classes, and hence we will
mostly refer to the fraction of deaths in our analysis instead
\end{comment}
{} For consistency, when we compared the statistics of a given variable
between two interventions, only simulations in which there was an
outbreak were considered. When it was needed to individuate if differences
between interventions were significant, we conducted statistical analysis:
t-- or Welch--tests for pairwise comparisons depending on whether
normality is fulfilled, and Tukey tests for multiple comparisons among
non-orthogonal variables.

\subsection*{Results}

When interventions are absent, the CFR takes values around 2\% when
we consider that individuals that would require hospitalization (but
not ICU) are recovered, raising up to \textasciitilde 11\% when we
considered that all would die %
\begin{comment}
APG: Verify these values
\end{comment}
. Unless otherwise stated, in the following we consider the latter
scenario to evaluate the interventions. In such scenario, the probability
of observing an outbreak is close to 0.85, having that 10\% of the
population in the camp would die and that \textasciitilde 83\% of
the population would recover, with the number of symptomatic cases
peaking after 55 days.

Self-distancing has a notable effect for the probability of outbreak,
with a 10\% decrease when just a reduction of a 20\% in the number
of contacts was considered (Fig. 2A). It was, however, needed a reduction
of up to 50\% to observe a more nuanced decrease in the fraction of
population dying, that could be reduced as much as a 35\% (Fig. 2B).
The same occurs for the time to the peak of symptomatics (Fig. 2C),
which increases up to 110 days, hence doubling the values when there
are no interventions in place. %
\begin{comment}
Recovered?
\end{comment}

Self-isolation brings a strong decrease close to a 30\% in the fraction
of population dying, when only 10 tents are available for a population
of 2000 individuals, i.e. a 0.5\% of the total population (Fig. 2).
Interestingly, increasing the number of tents does not improve much
this reduction and, when the number of tents represent a 10\% of the
population, the fraction of deaths even increases. This effect is
due to the assumption that each individual requires one carer (see
Supplementary Material). A side effect of this policy is that, when
the number of isolated population increases, it also increases the
number of healthy individuals in contact with the isolated population.
Still, the Case Fatality Rate experience a continuous, albeit small,
reduction with an increasing number of tents (Supplementary Fig. XX).
The probability of observing an outbreak and the time in whch the
symptomatic cases peak have a minimum (maximum) when \textasciitilde 1\%
of tents are considered (Fig. 2).

Importantly, these results hold when we assume that the time required
for individuals to recognize their symptoms is of 24h on average,
and the intervention becomes less effective when this time increases
(Supplementary Fig. XX). On the other hand, reducing this time to
12h does not significantly reduce the fraction of people dying %
\begin{comment}
This should be tested
\end{comment}
, but it does increase the time in which the number of symptomatic
cases peak (Supplementary Figs. XX and XX). If we additionally consider
that the symptomatic cases that would require hospitalization are
evacuated, it is only observed a 2\% reduction in the fraction of
individuals dying (Supplementary Fig. XXX). Since we assumed that
the evacuation does not change their fate, the intervention only affects
their infectivity. Although the period between developing more important
symptoms and dying is relatively long (\textasciitilde 10 days),
the number of individuals under these conditions is small, when compared
to the rest of the infected population.

Creating a safety zone improves overall the effect of previous interventions,
but sometimes with different outcomes for the exposed and protected
populations. For instance, the probability of outbreak decreases very
strongly (almost 40\%) for the protected population (see Fig. XX).
Notably, approximately 16\% of this reduction is due to the health
checks performed to get into the buffering zone, which exclude symptomatic
individuals (Suppl. Fig. XX). On the other hand, the probability of
outbreak may increase for the exposed population if the contacts per
week and per individual occuring in the buffering zone are reduced
to 2. This is a consequence of an increased number of contacts within
subpopulations, which may also explain the increase in the CFR of
the protected population (Supplementary Fig. XX), %
\begin{comment}
Understand this.
\end{comment}
. Despite of this side-effect, the fraction of individuals dying for
the protected subpopulation and, in turn, globally, is reduced. Another
important observation from this intervention is the notable increase
of the time to peak of symptomatics, raising up to 35\% globally,
and to 70\% for the protected population. Regarding the population
that should move to the safety zone, having only elderly or at most
elderly and adults with comorbidities, guarantee a low probability
of outbreak, which would otherwise increase the global fraction of
population dying (Supplementary Fig. XX). Positive effects of the
safety zone are even more marked for smaller population sizes, except
for the time to peak of symptomatic, which has a weaker effect for
smaller populations (Supplementary Fig. XXX and XXX).

The incorporation of a lockdown again shows the side-effect in which
if there is an outbreak in the safety zone, it has more negative effects,
due to the increased number of contacts of individuals within the
zone (Fig. XXX). However, the probability of outbreak is very strongly
reduced to a barely 10\% and, consequently, when there is an outbreak
in the camp (which in most of the cases will not reach the safe zone)
the fraction of deaths globally reduces (Fig. XXX).

Finally, the combination of interventions add their effects approximately
as it was shown when they were incorporated individually (Fig. XXX).
In particular, the creation of a safety zone improves the benefits
of the other interventions, and there are not noticeable side effects
beyond those previously described.

\section*{Discussion}

In this article, we proposed a number of feasible interventions of
immediate applicability, for human informal settlements. We took as
reference settlements of IDP in the NW of Syria, and we discussed
the feasibility of the different interventions with local stakeholders,
to individuate its economical viability and potential cultural issues.
In general, we considered worst case scenarios, possibly (hopefully)
leading to an overestimation of the number of casualties but that,
in this way, makes more apparent the potential of the interventions.
Our results are aligned with previous works XXX%
\begin{comment}
we should comment other results
\end{comment}

Self-distancing proves to be an efficient measure, and reducing the
number of contacts to a 50\% has the most important effect among all
the interventions. Such a reduction is, however, difficult to achieve,
for example due to the large population under 14 years old which have
high motility in the camps and little control.%
\begin{comment}
perhaps a little bit unrigurous the last point
\end{comment}

We proposed the implementation of self-isolation with individual tents,
that could be located either in a dedicated zone in the camp or next
to the tents of relatives, with a buffering zone ($\sim2\ m^{2})$
in-between. We think this intervention is more likely to be accepted
than the evacuation to isolation centers without health care {[}APG.
Include reference to public plans of WHO{]}. Without testing, it is
difficult to know if symptoms are truly associated to COVID-19, and
hence an increase in the contacts between individuals may increase
the infectivity, also among carers if they do not have effective protection
measures. Our proposal is effective even for a number of tents is
as low as 0.5-1\% of the total population in the camp. We considered
that there is one carer per individual isolated with a daily contact
in a buffering zone, and this choice shows that increasing the number
of isolated individuals increases the infectivity. This could be circumvented
with a more organized group of trained carers, which would reduce
the number of total contacts with infectious individuals. The intervention
is effective even though we assumed that individuals requiring hospitalization
should leave the tents due to their need of care, and hence they become
fully infectious.

The creation of a safety zone is an interesting intervention that
we propose should keep the same setting in terms of number of people
per tent and distance between tents. This would mitigate side-effects
derived from an increase in the number of contacts between vulnerable
population. Still, we implemented this increase exactly compensating
the reduction in the number of contacts with the other population,
and we observed in some cases side-effects such as an increase in
the fraction of deaths in the protected population, justifying these
concerns. However, these side-effects are compensated by the strong
reduction in the probability of outbreak in the safety zone that this
intervention provides, which leads to an overall reduction in the
fraction of deaths in the whole camp. Moreover, it is unrealistic
to think that the level of activity will be maintained in the safety
zone, which would readily control for any side-effects.

An important question for the success of this intervention comes from
the efficiency of the buffering zone. The reduction in the number
of contacts per week, the implementation of basic health checks to
identify symptomatic individuals, and the eventual lockdown, have
notable effects. Also important is which population should be shielded.
Protecting only elderly and at most adults with comorbidities have
the most beneficial effects, and the intervention becomes less effective
when more adults and kids are included. Another relevant consideration
is the fraction of population recovered, that may help to reach herd
immunity. We observed that unless a self-distancing of 50\% is considered,
the fraction of population recovered is close to a 75\%, which is
quite promising to prevent future outbreaks.

Decisions on which population should be protected are one example
of social challenges that may become more complex in practice. Another
example we did not address is that it is unlikely that a kid becomes
isolated alone in a tent. The number of exceptions to the cases proposed
are possibly endless, but we should ask if it is needed to run a simulation
to predict the outcome of these exceptions. In this respect, we believe
that the importance of these results comes from the fact that the
examples may help individuals to understand the dynamics of the virus,
and then to adapt to new situations and to look for alternatives.
Given the highly dynamic environment existing in the camps and the
slow reaction of the local and international authorities, empowering
the local population is possibly the best, if not the only way, to
help these communities.
\end{document}
