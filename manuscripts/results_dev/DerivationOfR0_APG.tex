%% LyX 2.3.2 created this file.  For more info, see http://www.lyx.org/.
%% Do not edit unless you really know what you are doing.
\documentclass{article}
\usepackage[utf8]{inputenc}
\usepackage{mathtools}
\usepackage{amsmath}

\makeatletter
%%%%%%%%%%%%%%%%%%%%%%%%%%%%%% User specified LaTeX commands.
\usepackage{xcolor}


\newcommand{\bs}[1]{\boldsymbol{#1}}
\newcommand{\add}[1]{\textcolor{red}{#1}}

\title{Derivation of R0}

\makeatother

\begin{document}
\maketitle

\subsection*{Reference and general idea}

The derivation of R0 for a SIR type model following description of
'Mathematical Tools for Understanding Infectious Disease Dynamics'
written by Odo Diekmann, Hans Heesterbeek and Tom Britton. See chapter
7, section 2 'Next-generation matrix for compartmental systems'. We
use 4 steps to find the basic reproduction number of a system. 1)
Define the infected subsystem, i.e. all equations that include compartments
where individuals can get infected. 2) Linearize the subsystem in
the disease free equilibrium. 3) Find the next generation matrix (NGM)
with large domain ($K_{L}$) by writing the linear system as: $\bs{\dot{x}}=(\bs{T}+\bs{\Sigma})\bs{x}$,
here $\bs{x}$ is a vector containing all states where individuals
can get infected, $\bs{T}$ contains all terms corresponding to transmission
and $\bs{\Sigma}$ all terms corresponding to transitions between
compartments. Then: $\bs{K}=-\bs{T}\bs{\Sigma}^{-1}.$ 4) The reproduction
number is now the dominant eigenvalue of the NGM. We also reference
the description of R0 in metapopulations in Philipps S, Rossi D. Mathematical
Models of Infectious Diseases: Two-Strain Infections in Metapopulations,
and the methodology for calculating R0 used by the authors in Gatto
M, Bertuzzo E, Mari L, Miccoli S, Carraro L, Casagrandi R, et al.
Spread and dynamics of the COVID-19 epidemic in Italy: Effects of
emergency containment measures. PNAS. 2020 May 12;117(19):10484--91.

\subsection*{Infected subsystem}

The infected subsystem is the following: \\
 
\begin{gather}
\dot{E}_{i}=\lambda_{i}S_{i}-\delta_{E}E_{i}\\
\dot{P}_{i}=\delta_{E}E_{i}-\delta_{p}P_{i}\\
\dot{A}_{i}=(1-f)\delta_{p}P_{i}-\gamma_{A}A_{i}\\
\dot{I}_{i}=f\delta_{p}P_{i}-\kappa_{i}I_{i}\\
\dot{H}_{i}=h_{i}\eta I_{i}-\gamma_{H}H_{i}
\end{gather}
\\

Where, \\
 
\begin{gather}
\lambda_{i}=\sum_{j=1}^{n}\beta_{ij}\frac{P_{j}+A_{j}+I_{j}+H_{j}}{N_{j}}\label{eq:lambda}
\end{gather}
\\

where $\beta_{ij}=\tau C_{ij}$ with $\tau$ being the probability
of infection for a contact with an infected person, and $C_{ij}$
is the average number of contacts of an individual of class $i$ with
an individual of class $j$.

\subsection*{Parametrization of the contact matrix}

We estimated the average number of contacts of individuals of class
$i$ in the camps, $\bar{c}_{i}$, and we parametrized the contact
matrix assuming that, in a well-mixed population, these contacts will
be distributed among classes relative to the fraction of individuals
within each class, i.e.

\begin{equation}
C_{ij}=\bar{c}_{i}N_{j}/N,
\end{equation}

with $N$ the total population size. A well-mixed population will
be considered the null model, and parameters derived under the null
model assumptions are indexed with the superscript $0$, e.g. the
null contact matrix is $C_{ij}^{0}$. The type of interventions that
we consider, aim to reduce either the average number of contacts a
class $i$ (e.g. self-isolation) or the accessibility of class $i$
to class $j$ (e.g. shielding strategies). We model the first type
of intervention introducing the parameter $\epsilon_{i}$, representing
the fraction of the average number of contacts observed in the null
model that prevail after the intervention: $\bar{c}_{i}=\epsilon_{i}\bar{c}_{i}^{0}$.
Similarly, we model the second type of intervention with the management
matrix $m_{ij}$, representing the fraction of population $j$ visible
to population $i$ after the intervention. The contact matrix resulting
from management strategies can therefore be written with respect to
the null model as:

\begin{equation}
C_{ij}=\epsilon_{i}m_{ij}\bar{c}_{i}^{0}N_{j}/N=\epsilon m_{ij}C_{ij}^{0}.
\end{equation}

Given this parameterization, we substitute in the expression for lambda
(Eq. \ref{eq:lambda}) to obtain for a model including management
strategies

\begin{gather}
\lambda_{i}=\frac{\tau}{N}\sum_{j=1}^{n}\epsilon_{i}\bar{c}_{i}^{0}m_{ij}\left(P_{j}+A_{j}+I_{j}+H_{j}\right)\label{eq:lambda_simplified}
\end{gather}


\subsection*{Estimation of the Next Generation Matrix}

We start considering the subsystem containing the infectious population
and, to facilitate notation, let us consider the following ordering
of the variables in the vector $x=(E_{1},...,E_{M},P_{1},...,P_{M},A_{1},...,A_{M},I_{1},...,I_{M},H_{1},...,H_{M})$,
with $M$ the number of population classes. We are interested in the
parametrization of the null model, which will serve as a baseline
to estimate the parameter $\tau$, which is unknown, and that does
not change when interventions are introduced. For the null model,
Eq. \ref{eq:lambda_simplified} becomes

\[
\lambda_{i}=\frac{\tau}{N}\sum_{j=1}^{n}\bar{c}_{i}^{0}\left(P_{j}+A_{j}+I_{j}+H_{j}\right).
\]

Following this notation, the linearized system can be written in the
form $\bs{\dot{x}}=(\bs{T}+\bs{\Sigma})\bs{x}$, where:

\begin{gather}
\bs{T}=\tau\begin{bmatrix}\bs{0} & \bs{\Theta} & \bs{\Theta} & \bs{\Theta} & \bs{\Theta}\\
\bs{0} & \bs{0} & \bs{0} & \bs{0} & \bs{0}\\
\bs{0} & \bs{0} & \bs{0} & \bs{0} & \bs{0}\\
\bs{0} & \bs{0} & \bs{0} & \bs{0} & \bs{0}\\
\bs{0} & \bs{0} & \bs{0} & \bs{0} & \bs{0}
\end{bmatrix}
\end{gather}

is the transmission matrix, with $\bs{\Theta}=\mathrm{diag}(p_{i}\bar{c}_{i}^{0})\bs{\bs{U}}$,
$p_{i}=N_{i}/N$, and $\bs{U}$ being the all-ones matrix of size
$M$. The transition matrix is

\begin{gather}
\bs{\Sigma}=\begin{bmatrix}-\delta_{E}\bs{I} & \bs{0} & \bs{0} & \bs{0} & \bs{0}\\
\delta_{E}\bs{I} & -\delta_{p}\bs{I} & \bs{0} & \bs{0} & \bs{0}\\
\bs{0} & (1-f)\delta_{p}\bs{I} & -\gamma_{A}\bs{I} & \bs{0} & \bs{0}\\
\bs{0} & f\delta_{p}\bs{I} & \bs{0} & -\mathrm{diag}(\kappa_{i})\bs{I} & \bs{0}\\
\bs{0} & \bs{0} & \bs{0} & \eta\mathrm{diag}(h_{i})\bs{I} & -\gamma_{H}\bs{I}
\end{bmatrix}
\end{gather}

Where $\bs{I}$ and $\bs{0}$ are the identity and null matrices of
size $M$, and $\kappa_{i}=((1-g_{i}-h_{i})\gamma_{I}+h_{i}\eta+g_{i}\alpha)$.
We next compute the inverse of the transition matrix

\begin{gather}
\bs{\Sigma^{-1}}=\begin{bmatrix}-\frac{1}{\delta_{E}}\bs{I} & \bs{0} & \bs{0} & \bs{0} & \bs{0}\\
-\frac{1}{\delta_{p}}\bs{I} & -\frac{1}{\delta_{p}}\bs{I} & \bs{0} & \bs{0} & \bs{0}\\
-\frac{(1-f)}{\gamma_{A}}\bs{I} & -\frac{(1-f)}{\gamma_{A}}\bs{I} & -\frac{1}{\gamma_{A}}\bs{I} & \bs{0} & \bs{0}\\
-f\mathrm{diag}(\frac{1}{\kappa_{i}})\bs{I} & -f\mathrm{diag}(\frac{1}{\kappa_{i}})\bs{I} & \bs{0} & -\mathrm{diag}(\frac{1}{\kappa_{i}})\bs{I} & \bs{0}\\
-\frac{f\eta}{\gamma_{H}}\mathrm{diag}(\frac{h_{i}}{\kappa_{i}})\bs{I} & -\frac{f\eta}{\gamma_{H}}\mathrm{diag}(\frac{h_{i}}{\kappa_{i}})\bs{I} & \bs{0} & -\frac{\eta}{\gamma_{H}}\mathrm{diag}(\frac{h_{i}}{\kappa_{i}})\bs{I} & -\frac{1}{\gamma_{H}}\bs{I}
\end{bmatrix}
\end{gather}

The NGM with large domain can now be found by $\bs{K_{\mathrm{L}}}=-\bs{T}\bs{\Sigma}^{-1}$,
however, as we know that each individual that gets infected first
will go to the $E$ compartment we can instead calculate the NGM with
small domain, ${\bf {K_{\mathrm{S}}}}$ that only consists of the
$E$ compartment \add{{[}Heffernan{]}}. We do this by removing from
$T$ the rows that correspond to the other compartments and from $\Sigma^{-1}$
the columns \add{APG: This could be more elegantly explained defining
an epsilon matrix} We then find:

\begin{gather*}
\bs{K_{\mathrm{S}}}=\tau\left[\left(\frac{1}{\delta_{P}}+\frac{(1-f)}{\gamma_{A}}\right)\bs{\Theta}+\mathrm{diag}\left(\frac{f}{\kappa_{i}}(1+\frac{h_{i}\eta}{\gamma_{H}})\right)\bs{\Theta}\right].
\end{gather*}


\subsection*{Estimation of $\tau$}

The reproduction number can be obtained from the maximum of the absolute
value of all eigenvalues of $\bs{K_{\mathrm{S}}}$, i.e. $R_{0}=|\lambda_{1}|$.
In our model all parameters are known except the probability of infection
per contact, $\tau$. We estimate the probability distribution of
$\tau$ generating realizations of $R_{0}$ and $\tilde{K_{\mathrm{S}}}=K_{\mathrm{S}}/\tau$
assuming a well mixed population as null model and solving for $\tau$:

\begin{equation}
\tau=\frac{R_{0}}{|\tilde{\lambda}_{1}^{0}|},
\end{equation}

where $|\tilde{\lambda}_{1}^{0}|$ is the maximum of the absolute
values of all eigenvalues of $\tilde{K_{\mathrm{S}}}$, when the contact
matrix is estimated under the null model, i.e. $\bar{c}_{i}=\bar{c}_{i}^{0}$.
\end{document}
