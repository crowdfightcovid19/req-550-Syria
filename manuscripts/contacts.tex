\documentclass{article}

%\usepackage{amsmath}
\usepackage{amsmath,amsfonts,amssymb,amsthm}
\usepackage{graphicx}
%\graphicspath{{fig/}}


\title{Contact matrix for camps}
\author{Req 550 Syria Team}

\begin{document}

\maketitle 

\section{Contact estimates}

Based on Chamsy's slide, the population distribution in each zone follows Table
\ref{tab:pop_distro}

\begin{table}[h!]
\begin{center}
\begin{tabular}{ |c|c|c| } 
 \hline
        & Orange        & Green \\ 
\hline
Young   & $\frac{50}{85}$ & 0 \\ 
\hline
Adult   & $\frac{35}{85}$ & $\frac{2}{3}$ \\ 
\hline
Elderly & 0               & $\frac{1}{3}$  \\ 
\hline
\end{tabular}
\caption{Fractions of the population in each age group}
\label{tab:pop_distro}
\end{center}
\end{table}

Note that age-groups in Chamsey's estimates differ slightly from our age
groups, therefore some of the fractions need to be adjusted.

Based on Chamsey's estimates (Slack), Table \ref{tab:contacts} summarises the
number of contacts per day for each age group.

\begin{table}[h!]
\begin{center}
\begin{tabular}{ |c|c| } 
 \hline
Age group & Contacts/day \\ 
\hline
Young     & 20 \\ 
\hline
Adult     & 10 \\ 
\hline
Elderly   & 10 \\ 
\hline
\end{tabular}
\caption{Average number of contacts per day for an individual in a given age
group.}
\label{tab:contacts}
\end{center}
\end{table}

\section{Contacts (perfect scenario)}
Since we lack detailed information on the contact structure, we assume contacts
for each age group are spread according to the population fractions. In the
perfect scenario (no infections happening in the neutral zone), Table
\ref{tab:contact_groups} shows the contacts between groups, per individual.

\begin{table}[h!]
\begin{center}
\begin{tabular}{|c|c|c|c|c|c|c| } 
 \hline
               & O/Y & O/A & O/E & G/Y & G/A & G/E \\ 
\hline
Orange/Young   & 11.76        & 8.24         & 0              & 0           & 0           & 0 \\ 
\hline
Orange/Adult   & 5.88         & 4.12         & 0              & 0           & 0           & 0 \\ 
\hline
Orange/Elderly & 0            & 0            & 0              & 0           & 0           & 0 \\ 
\hline
Green/Young    & 0            & 0            & 0              & 0           & 0           & 0 \\ 
\hline
Green/Adult    & 0            & 0            & 0              & 0           & 6.67        & 3.33 \\ 
\hline
Green/Elderly  & 0            & 0            & 0              & 0           & 6.67        & 3.33 \\ 
\hline
\end{tabular}
\caption{Contacts / day / individual, between groups.}
\label{tab:contact_groups}
\end{center}
\end{table}

\section{Contacts between zones}

In the proposed scenario, there is no direct contact between orange and green
zones. All the contacts happen through the neutral zone, that should have a
reduced probability of infection due to the distancing measures and the
personal protection equipment.

A conservative estimate for the interzonal contacts is the following. Assume
that half of the carers in the green zone (about 5\% of the green adult
population) will have daily contacts with 3 orange adults in the neutral zone,
to fulfill their caring duties (get products, discuss issues, etc).
Furthermore, assume that about 10\% of the green population will make use of
the neutral zone daily, have 3 contacts there with orange adults. Finally,
assume that effective (infectious) contacts in the neutral zone are only a fraction $\alpha$ of the
actual contacts. The direct interpretation of $\alpha$ is that only
$100\alpha\%$ of the meetings in the neutral zone include an actual contact,
but note that $\alpha$ depends in a non-linear manner with the
actual probability of infection. On the other hand, we assume that each person
of the green zone using the neutral zone will meet with three individuals of
the orange zone, distributed according to the fraction populations.

With this assumptions, we can update Table \ref{tab:contact_groups} to include
the interzonal contacts. Table \ref{tab:interzonal} contains the average
contacts per day, individual.

\begin{table}[h!]
\begin{center}
\begin{tabular}{|c|c|c|c|c|c|c| } 
 \hline
               & O/Y & O/A & O/E & G/Y & G/A & G/E \\ 
\hline
Orange/Young   & 11.76          & 8.24            & 0              & 0           & $0.0208\alpha$   & $0.01038\alpha$ \\ 
\hline
Orange/Adult   & 5.88           & 4.12            & 0              & 0           & $ 0.01501\alpha$ & $0.00727\alpha$ \\ 
\hline
Orange/Elderly & 0              & 0               & 0              & 0           & 0                & 0 \\ 
\hline
Green/Young    & 0              & 0               & 0              & 0           & 0                & 0 \\ 
\hline
Green/Adult    & $0.1176\alpha$ & $0.232\alpha$   & 0              & 0           & 6.67             & 3.33 \\ 
\hline
Green/Elderly  & $0.0588\alpha$ & $0.0412\alpha$  & 0              & 0           & 6.67             & 3.33 \\ 
\hline
\end{tabular}
\caption{Contacts / day / individual, between groups, including assumptions on
the usage of neutral zone.}
\label{tab:interzonal}
\end{center}
\end{table}

\section{How to use this information}

For each class of individuals X, from the expression for $R_0$ we will have an expression of
the form (with more terms!)
\[
    \beta_{Y_1} [\ldots] + \beta_{Y_2} [\ldots].
\]
Here, $\beta_Y$ is defined as $\kappa_{XY}\log(1-c)$, where $\kappa_{XY}$ is
the number of contacts per individual, per unit time (the values in the table)
from class $X$ with individuals of class $Y$. $c$ is the probability of
becoming infected in a contact. Assuming that all contacts are comparable, we can take the common factor 
$\log(1-c)$ and factorize the expression as
\[
    \log(1-c) \left( \kappa_{XY_1} [\ldots]+ \kappa_{XY_2} [\ldots] \right).
\]
Now, $\kappa$'s are known from Table \ref{tab:interzonal}, and with the
assumptions on $R_0$ we will be able to find $\log(1-c)$. For example,
$\kappa_{O/A,O/Y} = 5.88$.

\section{External contacts}
Using Chamsey's estimates, and denoting by $\iota_a,\iota_c$ the fraction of
infected individuals in the agricultural environment and the city environment,
and by $n_a,n_c$ the average number of external contacts of an orange adult working on
agriculture, or working in the city, we obtain that the external force of
infection for orange adults is
\[
0.0412 n_c \iota_c + 0.137 n_a \iota_a.
\]
We can assume that $n_c > n_a$ (more contacts in an urban environment).

\end{document}
